\section{Introduction}

Networks play a central role in online social media services like Twitter, Facebook, and Google+. These services allow a user to interact with others based on the online social network they curate through a process known as contact filtering~\cite{cazabet2012automated}. For example, `friends' on Facebook represent reciprocal links for sharing information, while `followers' on Twitter allow a single user to broadcast information in a one-to-many fashion. Central to all of these interactions is the fact that the \emph{structure} of the social network influences how information can be broadcast or diffuse through the service.

Because of the importance of structural networks in online social media, a large amount of work in this area has focused on using structural networks for community detection. Here, by `community' we mean the standard definition from the literature on social networks: a collection of nodes (users) within the network who are more highly connected to each other than to nodes (users) outside of the community~\cite{newman2004finding}. For instance, in~\cite{java2009we}, the authors use a follower network to determine communities within Twitter, and note that conversations tend to occur within these communities. The approach of focusing on structural networks makes sense for `real-world' sociological experiments, where obtaining additional information about user interactions may be expensive and time-consuming. However, with the prevalence of large, rich data sets for online social networks, additional information beyond the structure alone may be incorporated, and these augmented networks more realistically reflects how users interact with each other on social media services~\cite{grabowicz2012social}.

%A large body of work exists on methods for automatic detection of communities within networks~\cite{newman2004fast,newman2004finding,rosvall2008maps,blondel2008fast,LancichinettiPlos}. See~\cite{fortunato2010community} for a recent review. 
\textcolor{red}{A large body of work exists on methods for automatic detection of communities within networks, among which we recall the Girvan and Newman algorithm~\cite{newman2004finding}, Infomap~\cite{rosvall2008maps}, Louvain\cite{~blondel2008fast}, and the recently introduced OSLOM\cite{LancichinettiPlos}.}
All these methods \emph{begin} with a given network, and then attempt to uncover structure present in the network, i.e., they are agnostic to how the network was constructed. As opposed to this agnostic analysis, we propose and illustrate the importance of a question-focused approach. We believe that in order to understand the communities present in a data set, it is important to begin with a clear picture of the community type under consideration, and then perform the network collection and community detection with that community type in mind.

This is especially true for social network analysis. In online social networks, a `community' could refer to several possible structures. The simplest definition of community, as we have seen, might stem from the network of explicit connections between users on a service (friends, followers, etc.). On small time scales, these connections are more or less static, and we might instead determine communities based on who is talking to whom, providing a more dynamic picture. On a more abstract level, a user might consider themselves part of a community of people discussing similar topics. We might also define communities as collections of people who exhibit similar behaviors on a service, as in communities of teenagers vs.\ elderly users. We can characterize these types of communities based on the types of questions we might ask about them:
\begin{itemize}
	\item \textbf{Structure-based:} Who have you stated you are friends with? Who do you follow?
	\item \textbf{Activity-based:} Who do you act like?
	\item \textbf{Topic-based:} What do you talk about?
	\item \textbf{Interaction-based:} Who do you communicate with?
\end{itemize}

This is not meant to be an exhaustive list, but rather a list of some of the more common types of communities observed in social networks. We propose looking at when and how communities motivated by these different questions overlap, and whether different approaches to asking the question, ``What community are you in?'' leads to different insights about a social network. For example, a user on Twitter might connect mostly with computational social scientists, talk mostly about machine learning, interact solely with close friends (who may or may not be computational social scientists), and utilize the service only on nights and weekends. Each of these different `profiles' of the user highlight different views of the user's social network, and represent different types of communities. We divide our approaches into four categories based on the questions outlined above: structure-based, activity-based, topic-based, and interaction-based. The structure-based approach, as outlined above, is most common, and for our data relies on reported follower relationships.

The activity-based approach is motivated by the question of which individuals act in a homogeneous manner, e.g., which users use a service in a similar way. The main tools for answering this question stem from information theory. We consider each user on an online social network as an information processing unit, but ignore the content of their messages. 
In particular, our current activity-based approach was originally motivated by a methodology used to detect functional communities within populations of neurons~\cite{shalizi2007discovering}. Similar information theoretic approaches have been used with data arising from online social networks to gain insight into local user behavior~\cite{darmon2013understanding}, to detect communities based on undirected information flow~\cite{darmon2013detecting}, and to perform link detection~\cite{ver2012information}.

Our topic-based and interaction-based approaches, in contrast to the activity-based approach, rely on the \emph{content} of a user's interactions and ignore their temporal components. The content contains a great deal of information about the communication between users. For example, a popular approach to analyzing social media data is to use Latent Dirichlet Allocation (LDA) to infer topics based on the prevalence of words within a status~\cite{zhao2011comparing,michelson2010discovering}. The LDA model can then be used to infer distributions over latent topics, and the similarity of two users with respect to topics may be defined in terms of the distance between their associated topic distributions. Because our focus is not on topic identification, we apply a simpler approach using hashtags as a proxy for topics~\cite{becker2011beyond,tsur2012s}. We can then define the similarity of two users in terms of their hashtags, and use this similarity to build a topic-based network.

Finally, the interaction-based approach relies on the meta-data and text of messages to identify who a user converses with on the social media service. On Twitter, we can use mentions (indicating a directed communication) and \textcolor{blue}{retweets (indicating endorsement of another user)} to identify conversation. Moreover, we can define a directed influence between two users by considering the attention paid to that user compared to all other users. This allows us to generate a network based on conversations and user interactions.

The activity-based, topic-based, and interaction-based networks allow us to build a more complete picture of the \emph{implicit} social network present in online social media, as opposed to the \emph{explicit} social network indicated by structural links. In this paper, we explore the relation between these various possible networks and their corresponding communities. We begin by describing the methodologies used to generate the various types of networks, and infer their community structure. We then explore how the communities of users differ depending on the type of network used. Finally, we explore how communication patterns differ across and within the different community types.

\section{Related Work}

Previous research on communities in social networks focused almost exclusively on different network types in isolation. For example, an early paper considered the communities, and associated statistics, inferred from a follower network on Twitter~\cite{java2009we}. More recent work has considered the dynamics of communities based on structural links in Facebook~\cite{nguyen2011adaptive} and how structural communities impact mentions and retweets on Twitter~\cite{grabowicz2012social}.


Information theoretic, activity-based approaches have been applied previously to the analysis of networks arising in online social media~\cite{ver2012information,darmon2013detecting}, but to the best of our knowledge this is the first use of transfer entropy, an information theoretic measure of directed influence, for community detection.

For interaction-based communities, \cite{conover2011political} considered both mention and retweet networks in isolation for a collection of users chosen for their political orientation. In~\cite{deitrick2013mutually}, the authors construct a dynamic network based on simple time-windowed counts of mentions and retweets, and use the evolution of this network to aid in community detection.

There are two broad approaches to topic-based communities in the literature. \cite{rossi2012conversation} used a set of users collected based on their use of a single hashtag, and tracked the formation of follower and friendship links within that set of users. In~\cite{lim2012following}, the authors chose a set of topics to explore, and then seeded a network from a celebrity chosen to exemplify a particular topic. Both approaches thus begin with a particular topic in mind, and perform the data collection accordingly. Other approaches use probabilistic models for the topics and treat community membership as a latent variable~\cite{yin2012latent}.

% Structural:
% ==========

% \cite{java2009we} Determine friendship network on Twitter to define communities.

% \cite{nguyen2011adaptive} Uses a dynamic community detection algorithm (that uses snapshots of the network), but still focuses on \emph{structural} ties (Facebook friendship relationships)

% Interaction-based:
% ==================

% \cite{deitrick2013mutually} use 'dynamic' mention-retweet network to detect communities, but just augment the count of mentions retweets in the weighting (instead of using a proportion), and don't combine mentions and retweets.

% \cite{conover2011political} Retweet-mention analysis for political polarization analysis
% http://truthy.indiana.edu/site_media/pdfs/conover_icwsm2011_polarization.pdf

% Activity-based:
% ===============

% None, to the best of our knowledge.

% Topic-based:
% ============

% \cite{lim2012following} "We propose an efficient approach for detecting communities that share common interests on Twitter, based on linkages among followers of celebrities representing an interest category." Use both InfoMAP and clique percolation.

% \cite{rossi2012conversation} Collect users who use the same hashtag (a *single* hashtag, #XF5, for XFactor 5), and then note if this impacts the friend/follower network.

% \cite{xu2011sentiment} Generate 'sentiment' networks.

% Most-relevant:
% ==============

% \cite{lim2012tweets} This one is *very* important: they also note that structural links do not indicate communication. They use an interaction-based approach (mentions only). But they do not use a weighted version of the structural network. Instead, they use . And they do not go into much depth comparing how the two types of communities (structural and activity-based) differ. Moreover, they explicitly seed their network to focus around a particular topic (country music, tennis, basketball, etc.)

A notable exception to the analysis of isolated types of communities is~\cite{lim2012tweets}, which considered both structure-based and interaction-based communities on Twitter. However, this study focused on data collected based on particular topics (country music, tennis, and basketball), and not on a generic subpopulation of Twitter users. Moreover, it did not explore the differences in community structure resulting from the different network weightings, and focuses on aggregate statistics (community size, network statistics, etc.). Another notable exception is~\cite{kao2013talison}, where the authors used a tensor representation of user data to incorporate retweet and hashtag information into a study of the social media coverage of the Occupy Movement. The tensor can then be decomposed into factors in a generalization of the singular value decomposition of a matrix, and these factors can be used to determine `salient' users. However, this approach focused on data for a particular topic (the Occupy Movement) and did not collect users based on a structural network.