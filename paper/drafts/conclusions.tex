%File: formatting-instruction.tex
\documentclass[letterpaper]{article}
\usepackage{times}
\usepackage{helvet}
\usepackage{courier}
\usepackage{amssymb, amsmath, amsthm}
\usepackage{graphicx}
\usepackage{multirow}

\begin{document}

\section{Conclusion}

% What have we done:
% ==================

In this study, we have demonstrated that the communities observed in online social networks are highly question-dependent, and that different definitions of communities reveal different relationships between users. More importantly, we have shown that these different views of the network are not revealed by using the structural network alone. We found that community structure differs across community types on both the macro (e.g. number of communities and their size distribution) and micro (e.g. specific memberships, comemberships) scale.

We have also demonstrated that boundaries between communities represent meaningful internal/external divisions, with conversations and topics tending to be most highly concentrated within communities than without. We found this to be the case even when the communities were defined by a different criteria from the edge weights.

% * Defined four types of communities based natural questions one might ask about users in online social media.

% * Proposed three weightings to detect such communities.

% * Demonstrated that the community structure differed between weighting to weighting.

% Future Work:
% ============

% More with transfer entropy.
% 	Why did information tend to flow *across*, rather than within, community boundaries?

% Longitudinal study:
% 	How do the different types of communities change over time? Are activity-based communities more stable than interaction-based, etc.? For example, preliminary studies indicate that `conversations,' as defined by mentions and retweets, tend to move outside of structural communities over time.

Our work has introduced a novel use of transfer entropy for the detection of activity-based communities. In previous work on online social media~\cite{ver2012information}, transfer entropy was found useful for identifying influence between users. It is thus interesting that we find influence tends to be higher across community boundaries than within them. This counterintuitive result warrants further investigation. In addition, more rigorous choices of both the lag and time resolution based on methods for model selection should be explored~\cite{claeskens2008model}.

% Why Care? (i.e. impact):
% ========================

% What does this mean for researchers in online social media? Businesses?

% Using concepts like centrality to determine influence 
% 	* These metrics are based solely on the structural network.
% 	* Your *activity* may be more important than your *topology* in determining
% 	  your influence.

Our findings have important implications to a common problem in social network analysis: identification of influential individuals. Many network measures of influence are based on the various types of centrality (degree, betweenness, closeness, eigenvector, etc.)~\cite{newman2009networks}. Most centralities depend explicitly on the structure of the network under consideration. But we have seen in our study that a structural network is not sufficient to capture user interaction in online social media. Thus, a naive application of centrality measures to a structural network for influence detection may give rise to erroneous results. This result has been explored previously~\cite{kitsak2010identification}, and our work further highlights its importance. Moreover, weighted generalizations of these centralities might lead to better insights about who is actually influential in an online social network.

% More generally, our work demonstrates that asking the proper question is an unavoidable first step to community detection, and that the rich data sets available from 

\end{document}